\setcounter{chapter}{2}

\section{MariaDB mit OQGRAPH} %Kapitel: Graph-Datenbanken - Grundlegende technologische Aspekte (chapter)
\subsection{Installation}
In diesem Kapitel wird die Installation und Konfiguration des DBS MaraiDB, sowie des Plugins QOGRAPH auf einem Ubuntu Rechner beschrieben.  
QOGRAPH wird als Plugin in einem eigenständigem Softwarepaket verteilt. Es ist nicht zwingend nötig den MariaDB-Server explizit zu installieren, dieser ist als Abhängigkeit definiert und wird automatisch mit installiert, wenn keine gültige Installation gefunden wird. Alle benötigten Pakete können mittels apt-get installiert werden.
\begin{lstlisting}
sudo apt-get install mariadb-plugin-oqgraph
\end{lstlisting}
Danach muss das Plugin noch installiert werden. Dafür in die DBMS-Konsole \texttt{mysql} wechseln und folgendes Kommando eingeben. Die Installation ist damit abgeschlossen.
\begin{lstlisting}
INSTALL SONAME 'ha_oqgraph';
\end{lstlisting}

Dabei ist zu beachten, dass MaraiDB das User-Directory von Unix nutzt. Dementsprechend kann nur mit root Rechten in die \texttt{mysql} Konsole  gewechselt werden. Es ist also nötig \texttt{sudo} zu verwenden. Damit auch normale User die Datenbank administrieren können, muss das \texttt{unix\_socket} Plugin deaktiviert werden.

\subsection{Grundlegende Funktionsweise}
\subsection{Aufbau}
