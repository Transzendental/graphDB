\chapter{MariaDB mit OQGRAPH}
\section{Installation}
In diesem Kapitel wird die Installation und Konfiguration des DBS MaraiDB, sowie des Plugins QOGRAPH auf einem Ubuntu Rechner beschrieben.  
QOGRAPH wird als Plugin in einem eigenständigem Softwarepaket verteilt. Es ist nicht zwingend nötig den MariaDB-Server explizit zu installieren, dieser ist als Abhängigkeit definiert und wird automatisch mit installiert, wenn keine gültige Installation gefunden wird. Alle benötigten Pakete können mittels apt-get installiert werden.
\begin{lstlisting}
sudo apt-get install mariadb-plugin-oqgraph
\end{lstlisting}
Danach muss das Plugin noch installiert werden. Dafür in die DBMS-Konsole \texttt{mysql} wechseln und folgendes Kommando eingeben. %TODO Rechte mysql konsole
\begin{lstlisting}
INSTALL SONAME 'ha_oqgraph';
\end{lstlisting}

\section{Grundlegende Funktionsweise}
\section{Aufbau}
