\subsection{Einrichtung des Grundschemas}
OQGRAPH wird als Engine beschrieben, welche dem Nutzer die Handhabung hierarchische Strukturen (Bäume) und komplexe Graphen (zahlreiche Kanten) ermöglichen soll \cite{oqgraph}. Dieser Aussage wird OQGRAPH, insbesondere im erstem Punkt, nicht gerecht. 

% Der Grund dafür ... nur kanten gespeichert ...

%- key auf node-IDs in relationaler Knoten tabelle

\subsection{Use Cases}
\subsubsection{CREATE}
\subsubsection{READ}
\subsubsection{UPDATE}
\subsubsection{DELETE}
\subsection{Umsetzung eines PHP-Clients}
\subsection{Fazit}
